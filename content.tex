
\begin{section}

 \chapter{Concepts fondamentaux de MIR}
 %  \addcontentsline{toc}{chapter}{Concepts fondamentaux de MIR}
 Les concepts fondamentaux de la Récupération d'Information basée sur la
 Recherche de Médias (MIR -- Multimedia Information Retrieval) englobent
 plusieurs notions clés qui permettent de rechercher et de récupérer des
 informations multimédia (textes, images, audio, vidéo, etc.) de manière
 efficace. Voici un aperçu des concepts essentiels dans ce domaine:
 \setcounter{section}{1}
 \section{Indexation multimodale}
 %  \addcontentsline{toc}{section}{Indexation multimodale}

 L'indexation est le processus d'attribution de métadonnées aux objets
 multimédia afin qu'ils puissent être rapidement récupérés lors d'une requête.
 L'indexation multimodale fait référence à l'organisation de différents types de
 médias (textes, images, vidéos, audio) de manière à permettre une recherche
 efficace sur l'ensemble de ces données.

 \subsection{Indexation textuelle}
 %  \addcontentsline{toc}{subsection}{Indexation textuelle}
 L’indexation textuelle consiste à associer des mots-clés ou des métadonnées à
 des documents textuels pour en faciliter la recherche et l’organisation. Elle
 repose sur l’identification des termes significatifs ou expressions clés qui
 représentent le contenu, tels que énergie solaire ou transition énergétique
 pour un article sur les énergies renouvelables. Les métadonnées peuvent inclure
 des informations comme l’auteur, la date de création, les catégories
 thématiques ou un résumé descriptif. L’indexation peut être effectuée
 manuellement par des experts ou automatiquement à l’aide d’algorithmes
 utilisant des techniques de traitement automatique du langage naturel (TALN).
 Ce processus améliore considérablement l’accès à l’information, permet de
 retrouver rapidement des documents pertinents dans de grandes bases de données
 et optimise les moteurs de recherche. Par exemple, pour un document intitulé
 "L'intelligence artificielle dans la médecine moderne", les mots-clés
 pourraient inclure IA, médecine, ou diagnostic assisté par ordinateur.
 \subsection{Indexation visuelle}
 %  \addcontentsline{toc}{subsection}{Indexation visuelle}

 L’indexation visuelle consiste à extraire et analyser les caractéristiques
 visuelles des images et vidéos pour les organiser, les classifier et faciliter
 leur recherche. Ces caractéristiques peuvent inclure des informations telles
 que les couleurs dominantes, les textures, les formes ou encore les objets
 reconnus au sein des images. Par exemple, une image contenant un coucher de
 soleil pourra être associée à des teintes chaudes (orange, rouge), des textures
 de ciel et des formes naturelles comme des montagnes. L’indexation visuelle
 repose souvent sur des algorithmes d’apprentissage automatique ou de vision par
 ordinateur qui permettent d’identifier et d’étiqueter automatiquement les
 contenus visuels. Ce processus est largement utilisé dans des domaines tels que
 les bibliothèques d’images, les moteurs de recherche visuelle, ou encore les
 systèmes de surveillance, permettant une navigation intuitive et efficace parmi
 des collections visuelles volumineuses.
 \subsection{Indexation audio}
 %  \addcontentsline{toc}{subsection}{Indexation audio}
 L’indexation audio implique l’identification et l’extraction de
 caractéristiques spécifiques des fichiers audio pour en faciliter
 l’organisation et la recherche. Parmi ces caractéristiques, on trouve les
 fréquences, qui permettent d’identifier les tonalités ou les gammes, et les
 rythmes, qui aident à décrire le tempo et la structure temporelle du son.
 D’autres éléments comme les timbres (ou les qualités sonores) et les motifs
 musicaux peuvent également être analysés pour mieux comprendre le contenu
 audio. Cette indexation est souvent effectuée à l’aide d’algorithmes de
 traitement du signal et d’apprentissage automatique qui reconnaissent des
 schémas audio distinctifs, facilitant ainsi la catégorisation de fichiers
 musicaux, les recherches par contenu, ou l’identification d’échantillons
 sonores dans de grandes bases de données.

 \section{Recherche d'informations multimédia}
 %  \addcontentsline{toc}{section}{Recherche d'informations multimédia}
 Dans MIR, les informations sont représentées sous des formats spécifiques pour
 faciliter leur récupération. Cela inclut la représentation de données
 multimédia sous forme de vecteurs ou de signatures, souvent avec des
 caractéristiques extraites des éléments multimédias.
 \subsection{Modèle vectoriel}
 Le modèle vectoriel pour la recherche d'informations multimédia représente les
 documents et les requêtes sous forme de vecteurs dans un espace
 multidimensionnel. Chaque dimension correspond à une caractéristique (termes,
 couleurs, fréquences, etc.), et la pertinence est calculée en mesurant la
 similarité, souvent via le cosinus de l'angle entre les vecteurs. Ce modèle est
 largement utilisé pour classer et retrouver des contenus en fonction de leur
 similarité avec la requête
 \subsection{Modèle probabiliste}
 Il repose sur l'idée d'estimer la probabilité qu'un document soit pertinent
 pour une requête spécifique. Chaque document est représenté par un ensemble de
 caractéristiques (termes textuels, éléments visuels ou audio), et un score de
 pertinence est calculé en fonction de ces caractéristiques et de leur
 importance pour l'utilisateur. Les documents sont ensuite classés en ordre
 décroissant de probabilité pour retourner les résultats les plus pertinents en
 priorité. Ce modèle s'adapte aux incertitudes inhérentes à la recherche et est
 souvent amélioré par des techniques comme le feedback de pertinence ou
 l'apprentissage automatique. Il est particulièrement efficace pour intégrer
 différents types de données multimodales. \vfill

\end{section}
