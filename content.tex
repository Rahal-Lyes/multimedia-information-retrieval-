
\begin{section}

 \chapter{Concepts fondamentaux de MIR}
 % \addcontentsline{toc}{chapter}{Concepts fondamentaux de MIR}

 Les concepts fondamentaux de la Récupération d'Information basée sur la
 Recherche de Médias (MIR -- Multimedia Information Retrieval) englobent
 plusieurs notions clés qui permettent de rechercher et de récupérer des
 informations multimédia (textes, images, audio, vidéo, etc.) de manière
 efficace. Voici un aperçu des concepts essentiels dans ce domaine :

 \section{Indexation multimodale}
 % \addcontentsline{toc}{section}{Indexation multimodale}

 L'indexation est le processus d'attribution de métadonnées aux objets
 multimédia afin qu'ils puissent être rapidement récupérés lors d'une requête.
 L'indexation multimodale fait référence à l'organisation de différents types de
 médias (textes, images, vidéos, audio) de manière à permettre une recherche
 efficace sur l'ensemble de ces données.

 \subsection{Indexation textuelle}
 % \addcontentsline{toc}{subsection}{Indexation textuelle}

 L’indexation textuelle consiste à associer des mots-clés ou des métadonnées à
 des documents textuels pour en faciliter la recherche et l’organisation. Elle
 repose sur l’identification des termes significatifs ou expressions clés qui
 représentent le contenu, tels que \emph{énergie solaire} ou \emph{transition
   énergétique} pour un article sur les énergies renouvelables. Les métadonnées
 peuvent inclure des informations comme l’auteur, la date de création, les
 catégories thématiques ou un résumé descriptif. L’indexation peut être
 effectuée manuellement par des experts ou automatiquement à l’aide
 d’algorithmes utilisant des techniques de traitement automatique du langage
 naturel (TALN). Par exemple, pour un document intitulé \emph{L'intelligence
   artificielle dans la médecine moderne}, les mots-clés pourraient inclure
 \emph{IA}, \emph{médecine}, ou \emph{diagnostic assisté par ordinateur}.

 \subsection{Indexation visuelle}
 % \addcontentsline{toc}{subsection}{Indexation visuelle}

 L’indexation visuelle consiste à extraire et analyser les caractéristiques
 visuelles des images et vidéos pour les organiser, les classifier et faciliter
 leur recherche. Ces caractéristiques peuvent inclure des informations telles
 que les couleurs dominantes, les textures, les formes ou encore les objets
 reconnus au sein des images. Par exemple, une image contenant un coucher de
 soleil pourra être associée à des teintes chaudes (orange, rouge), des textures
 de ciel et des formes naturelles comme des montagnes. L’indexation visuelle
 repose souvent sur des algorithmes d’apprentissage automatique ou de vision par
 ordinateur.

 \subsection{Indexation audio}
 % \addcontentsline{toc}{subsection}{Indexation audio}

 L’indexation audio implique l’identification et l’extraction de
 caractéristiques spécifiques des fichiers audio pour en faciliter
 l’organisation et la recherche. Parmi ces caractéristiques, on trouve les
 fréquences, les rythmes, les timbres et les motifs musicaux. Cette indexation
 est souvent effectuée à l’aide d’algorithmes de traitement du signal et
 d’apprentissage automatique, facilitant ainsi la catégorisation de fichiers
 musicaux et les recherches par contenu.

 \section{Recherche d'informations multimédia}
 % \addcontentsline{toc}{section}{Recherche d'informations multimédia}

 Dans MIR, les informations sont représentées sous des formats spécifiques pour
 faciliter leur récupération. Cela inclut la représentation de données
 multimédia sous forme de vecteurs ou de signatures, souvent avec des
 caractéristiques extraites des éléments multimédias.

 \subsection{Modèle vectoriel}

 Le modèle vectoriel pour la recherche d'informations multimédia représente les
 documents et les requêtes sous forme de vecteurs dans un espace
 multidimensionnel. Chaque dimension correspond à une caractéristique (termes,
 couleurs, fréquences, etc.), et la pertinence est calculée en mesurant la
 similarité, souvent via le cosinus de l'angle entre les vecteurs.

 \subsection{Modèle probabiliste}

 Le modèle probabiliste repose sur l'idée d'estimer la probabilité qu'un
 document soit pertinent pour une requête spécifique. Chaque document est
 représenté par un ensemble de caractéristiques (termes textuels, éléments
 visuels ou audio), et un score de pertinence est calculé. Ce modèle s'adapte
 aux incertitudes inhérentes à la recherche et est souvent amélioré par des
 techniques comme le feedback de pertinence ou l'apprentissage automatique.

 \section{Requête multimodale}
 % \addcontentsline{toc}{section}{Requête multimodale}

 Les requêtes multimodales dans la Recherche d'Informations Multimodales (MIR)
 sont des requêtes où plusieurs types de médias ou modalités sont utilisés
 ensemble pour améliorer la précision des résultats de recherche. Ces modalités
 peuvent inclure du texte, des images, des vidéos, des éléments audio, des
 gestes, ou d'autres types de données.

 \subsection{Processus de gestion des requêtes multimodales}

 \paragraph{Fusion des modalités :} Lorsqu'une requête multimodale est lancée, les différents types de médias sont
 traités et intégrés de manière cohérente. Par exemple, dans le cas d’une
 requête comprenant un texte et une image, un système peut utiliser des
 techniques de fusion de caractéristiques, où des vecteurs de caractéristiques
 représentant le texte et l'image sont combinés.

 \paragraph{Alignement sémantique :} L'un des défis majeurs des requêtes multimodales est d'assurer un alignement
 sémantique entre les différentes modalités. Par exemple, une image peut
 contenir des objets ou des scènes qui ne sont pas directement exprimés par des
 mots dans une requête textuelle. Des techniques avancées comme les réseaux
 neuronaux multimodaux sont utilisées pour apprendre les relations entre les
 différentes modalités.

 \paragraph{Recherche basée sur des requêtes combinées :} Un utilisateur peut effectuer une recherche multimodale en combinant un texte
 descriptif avec une image ou une vidéo. Par exemple, une recherche pourrait
 combiner un texte comme \emph{montagnes au coucher du soleil} et une image
 représentant un paysage montagneux. Le système doit analyser à la fois le texte
 pour en extraire les mots-clés et l'image pour en détecter les éléments visuels
 pertinents.

\end{section}
