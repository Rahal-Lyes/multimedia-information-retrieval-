
\begin{section}

  \chapter{Concepts fondamentaux de MIR}
 %  \addcontentsline{toc}{chapter}{Concepts fondamentaux de MIR}
 Les concepts fondamentaux de la Récupération d'Information basée sur la
 Recherche de Médias (MIR -- Multimedia Information Retrieval) englobent
 plusieurs notions clés qui permettent de rechercher et de récupérer des
 informations multimédia (textes, images, audio, vidéo, etc.) de manière
 efficace. Voici un aperçu des concepts essentiels dans ce domaine:
 \setcounter{section}{1}
 \section{Indexation multimodale}
 %  \addcontentsline{toc}{section}{Indexation multimodale}

 L'indexation est le processus d'attribution de métadonnées aux objets
 multimédia afin qu'ils puissent être rapidement récupérés lors d'une requête.
 L'indexation multimodale fait référence à l'organisation de différents types de
 médias (textes, images, vidéos, audio) de manière à permettre une recherche
 efficace sur l'ensemble de ces données.

 \subsection{Indexation textuelle}
 %  \addcontentsline{toc}{subsection}{Indexation textuelle}
 L’indexation textuelle consiste à associer des mots-clés ou des métadonnées à
 des documents textuels pour en faciliter la recherche et l’organisation. Elle
 repose sur l’identification des termes significatifs ou expressions clés qui
 représentent le contenu, tels que énergie solaire ou transition énergétique
 pour un article sur les énergies renouvelables. Les métadonnées peuvent inclure
 des informations comme l’auteur, la date de création, les catégories
 thématiques ou un résumé descriptif. L’indexation peut être effectuée
 manuellement par des experts ou automatiquement à l’aide d’algorithmes
 utilisant des techniques de traitement automatique du langage naturel (TALN).
 Ce processus améliore considérablement l’accès à l’information, permet de
 retrouver rapidement des documents pertinents dans de grandes bases de données
 et optimise les moteurs de recherche. Par exemple, pour un document intitulé
 "L'intelligence artificielle dans la médecine moderne", les mots-clés
 pourraient inclure IA, médecine, ou diagnostic assisté par ordinateur.
 \subsection{Indexation visuelle}
 %  \addcontentsline{toc}{subsection}{Indexation visuelle}

 L’indexation visuelle consiste à extraire et analyser les caractéristiques
 visuelles des images et vidéos pour les organiser, les classifier et faciliter
 leur recherche. Ces caractéristiques peuvent inclure des informations telles
 que les couleurs dominantes, les textures, les formes ou encore les objets
 reconnus au sein des images. Par exemple, une image contenant un coucher de
 soleil pourra être associée à des teintes chaudes (orange, rouge), des textures
 de ciel et des formes naturelles comme des montagnes. L’indexation visuelle
 repose souvent sur des algorithmes d’apprentissage automatique ou de vision par
 ordinateur qui permettent d’identifier et d’étiqueter automatiquement les
 contenus visuels. Ce processus est largement utilisé dans des domaines tels que
 les bibliothèques d’images, les moteurs de recherche visuelle, ou encore les
 systèmes de surveillance, permettant une navigation intuitive et efficace parmi
 des collections visuelles volumineuses.
 \subsection{Indexation audio}
 %  \addcontentsline{toc}{subsection}{Indexation audio}
 L’indexation audio implique l’identification et l’extraction de
 caractéristiques spécifiques des fichiers audio pour en faciliter
 l’organisation et la recherche. Parmi ces caractéristiques, on trouve les
 fréquences, qui permettent d’identifier les tonalités ou les gammes, et les
 rythmes, qui aident à décrire le tempo et la structure temporelle du son.
 D’autres éléments comme les timbres (ou les qualités sonores) et les motifs
 musicaux peuvent également être analysés pour mieux comprendre le contenu
 audio. Cette indexation est souvent effectuée à l’aide d’algorithmes de
 traitement du signal et d’apprentissage automatique qui reconnaissent des
 schémas audio distinctifs, facilitant ainsi la catégorisation de fichiers
 musicaux, les recherches par contenu, ou l’identification d’échantillons
 sonores dans de grandes bases de données.

 \section{Recherche d'informations multimédia}
 %  \addcontentsline{toc}{section}{Recherche d'informations multimédia}
 Dans MIR, les informations sont représentées sous des formats spécifiques pour
 faciliter leur récupération. Cela inclut la représentation de données
 multimédia sous forme de vecteurs ou de signatures, souvent avec des
 caractéristiques extraites des éléments multimédias.
 \subsection{Modèle vectoriel}
 Le modèle vectoriel pour la recherche d'informations multimédia représente les
 documents et les requêtes sous forme de vecteurs dans un espace
 multidimensionnel. Chaque dimension correspond à une caractéristique (termes,
 couleurs, fréquences, etc.), et la pertinence est calculée en mesurant la
 similarité, souvent via le cosinus de l'angle entre les vecteurs. Ce modèle est
 largement utilisé pour classer et retrouver des contenus en fonction de leur
 similarité avec la requête
 \subsection{Modèle probabiliste}
 Il repose sur l'idée d'estimer la probabilité qu'un document soit pertinent
 pour une requête spécifique. Chaque document est représenté par un ensemble de
 caractéristiques (termes textuels, éléments visuels ou audio), et un score de
 pertinence est calculé en fonction de ces caractéristiques et de leur
 importance pour l'utilisateur. Les documents sont ensuite classés en ordre
 décroissant de probabilité pour retourner les résultats les plus pertinents en
 priorité. Ce modèle s'adapte aux incertitudes inhérentes à la recherche et est
 souvent amélioré par des techniques comme le feedback de pertinence ou
 l'apprentissage automatique. Il est particulièrement efficace pour intégrer
 différents types de données multimodales.
 \section{Requête multimodale}
 Les requêtes multimodales dans la Recherche d'Informations Multimodales (MIR)
 sont des requêtes où plusieurs types de médias ou modalités sont utilisés
 ensemble pour améliorer la précision des résultats de recherche. Ces modalités
 peuvent inclure du texte, des images, des vidéos, des éléments audio, des
 gestes, ou d'autres types de données. Ce type de recherche est particulièrement
 utile dans des contextes où une seule modalité, comme le texte ou l'image, ne
 permet pas de capturer suffisamment d'informations pour retourner des résultats
 pertinents
 \subsection{Processus de gestion des requêtes multimodales}

 \textbf{1 -Fusion des modalités :} Lorsqu'une requête multimodale est lancée, les différents types de médias sont
 traités et intégrés de manière cohérente. Par exemple, dans le cas d’une
 requête comprenant un texte et une image, un système peut utiliser des
 techniques de fusion de caractéristiques, où des vecteurs de caractéristiques
 représentant le texte et l'image sont combinés. Ensuite, la similarité entre la
 requête et les documents dans la base de données est calculée sur la base de
 cette fusion. De même, pour des requêtes qui incluent à la fois des vidéos et
 des éléments audio, des modèles de traitement multimodal peuvent être utilisés
 pour extraire des caractéristiques pertinentes de chaque modalité (par exemple,
 des objets détectés dans la vidéo et des fréquences spécifiques dans l'audio)
 et pour les intégrer dans un espace de recherche commun.
 \par
 \textbf{2 -Alignement sémantique :}L'un des défis majeurs des requêtes multimodales est d'assurer un alignement
 sémantique entre les différentes modalités. Par exemple, une image peut contenir des objets ou des scènes qui
 ne sont pas directement exprimés par des mots dans une requête textuelle, et vice versa. Des techniques avancées
 comme les réseaux neuronaux multimodaux sont utilisées pour apprendre les relations entre les différentes modalités
 (par exemple, associant un objet visuel à des mots-clés dans une requête textuelle).
 Le but est de s'assurer que les informations provenant de différentes modalités sont correctement interprétées pour
 que le système puisse comprendre ce que l'utilisateur recherche, même si les termes textuels ne correspondent
 pas exactement aux éléments visuels.

 \par
 \textbf{3 -Recherche basée sur des requêtes combinées :}
 Par exemple, un utilisateur peut effectuer une recherche multimodale en combinant un texte descriptif avec une
 image ou une vidéo. Un exemple concret serait une recherche dans une base de données d'images où l'utilisateur
 soumet un texte comme « montagnes au coucher du soleil » et une image représentant un paysage montagneux.
 Le système doit être capable d'analyser à la fois le texte pour en extraire les mots-clés (comme "montagnes" et
 "coucher du soleil") et l'image pour en détecter les éléments visuels pertinents (comme les montagnes et les
 couleurs chaudes). Ensuite, il associera ces informations pour retourner des résultats correspondant aux deux
 modalités.

 \par

 \textbf{4 -Traitement des données audio et vidéo :}
 Dans des cas plus complexes, comme la recherche multimodale dans des vidéos, la requête peut inclure à la fois du
 contenu visuel et audio. Par exemple, un utilisateur pourrait soumettre un extrait audio d'une chanson et une image
 d'un artiste pour retrouver des vidéos associées à cette chanson ou à cet artiste. Le système devra traiter le son
 pour identifier des motifs ou des fréquences musicales, et analyser l'image pour en extraire des éléments visuels
 tels que des visages ou des logos, avant de faire correspondre les deux types de données à des vidéos existantes
 dans la base de données.

%  \section{Requête par contenu (Content-Based Retrieval)}
%  La requête par contenu (Content-Based Retrieval, ou CBR) est une approche où
%  l'utilisateur spécifie des critères de recherche directement basés sur le
%  contenu multimédia, plutôt que de s'appuyer sur des métadonnées associées comme
%  des titres, des mots-clés ou des descriptions. L'objectif est de retrouver des
%  objets multimédia similaires en fonction de leurs caractéristiques
%  intrinsèques, telles que les couleurs, les textures, les formes, ou encore les
%  sons.
%  \subsection{Recherche d'images}
%  La recherche d'images basée sur le contenu consiste à trouver des images
%  similaires à une image donnée. Au lieu d'utiliser des métadonnées comme des
%  mots-clés ou des tags, cette approche se concentre sur des descripteurs visuels
%  extraits directement de l'image. \par \textbf{1 -Les couleurs dominantes:}
%  La recherche peut être effectuée en comparant les histogrammes de couleurs ou
%  les caractéristiques de couleur (par exemple, les couleurs dominantes dans une
%  image). \par \textbf{2 -Les textures :}Les techniques de reconnaissance de
%  textures utilisent des algorithmes comme les matrices de co-occurrence pour
%  identifier des motifs visuels similaires.
%  \par \textbf{3 -Les formes :}
%  Les descripteurs de formes comme les contours ou les régions homogènes dans une image peuvent être utilisés pour
%  rechercher des images similaires en termes de structure visuelle. L'objectif est de permettre à l'utilisateur de
%  soumettre une image et de trouver d'autres images qui présentent des caractéristiques visuelles similaires,
%  facilitant ainsi la recherche par exemple de produits, de paysages ou de visuels similaires.
%  \subsection{Recherche vidéo}
%  La recherche vidéo basée sur le contenu s’appuie sur des extraits vidéo ou des
%  éléments spécifiques au sein des vidéos. Plutôt que de se fier uniquement aux
%  métadonnées (comme le titre de la vidéo), cette approche recherche dans le
%  contenu vidéo lui-même.

%  \par \textbf{1 -Les objets ou scènes dans les vidéos :}L’analyse des objets présents dans les vidéos (par exemple, un visage ou un véhicule) ou des scènes (comme un paysage ou un intérieur) peut être effectuée grâce à des techniques de vision par ordinateur, telles que la détection d'objets.
%  \par \textbf{2 -Les mouvements et transitions :}Le suivi de mouvements ou d'actions au sein de la vidéo permet d'identifier des vidéos contenant des séquences similaires, même si elles diffèrent par d'autres aspects.
%  \par \textbf{3 -Les métadonnées visuelles et audio :}La recherche vidéo peut également s’appuyer sur des caractéristiques audio (comme des dialogues ou des effets sonores) et visuelles (comme des changements de scènes ou des objets détectés). Cela permet une recherche plus précise en fonction du contenu réel de la vidéo, permettant par exemple de retrouver des vidéos présentant des actions spécifiques ou des scènes similaires.
%  \subsection{Recherche audio}
%  La recherche dans les fichiers audio basée sur le contenu consiste à analyser
%  les caractéristiques sonores du fichier audio pour le comparer à la requête de
%  l'utilisateur.
%  \par \textbf{1 -La fréquence :}L'analyse de la fréquence permet de détecter des éléments comme la hauteur, les gammes ou les tonalités dans un fichier audio. La recherche peut être effectuée en comparant les spectres de fréquence des fichiers audio.
%  \par \textbf{2 -Le rythme et la structure temporelle:}Pour la musique, la recherche peut être basée sur des motifs rythmiques ou des structures musicales spécifiques. Cela permet de retrouver des morceaux qui partagent un rythme similaire, même si le reste de la composition est différent.
%  \par \textbf{3 -Les motifs audio spécifiques :}L'extraction de motifs récurrents ou de signatures sonores peut être utilisée pour rechercher des éléments audio particuliers dans de grands ensembles de données sonores (comme des discours, des effets sonores, ou des extraits musicaux).

 \section{Mesure de la performance}
 La mesure de la performance dans les systèmes de recherche d'informations multimédia (MIR) est cruciale pour évaluer l'efficacité et la pertinence des résultats retournés. Ces systèmes cherchent à fournir des résultats multimodaux (texte, image, audio, vidéo) en réponse à une requête donnée
 \subsection{Précision}
 La précision est une mesure qui indique la proportion de documents ou objets
 pertinents parmi les résultats retournés par le système de recherche. Elle est
 exprimée par la formule suivante :

 \[
   \text{Précision} = \frac{\text{Nombre de documents pertinents récupérés}}{\text{Nombre total de documents récupérés}}
   \vspace{1cm}
 \]
 En d'autres termes, elle mesure combien de résultats récupérés par le système
 sont réellement pertinents par rapport à l'ensemble des résultats retournés.
 Une précision élevée signifie que le système fournit principalement des
 résultats pertinents, mais cela ne garantit pas nécessairement que tous les
 documents pertinents ont été trouvés.

 \subsection{Rappel}
 Le rappel, également connu sous le nom de sensibilité ou de couverture, est une
 mesure qui évalue la capacité du système à retrouver tous les documents
 pertinents disponibles dans la base de données. Il est défini comme la
 proportion de documents pertinents récupérés parmi tous les documents
 pertinents existants. La formule du rappel est la suivante :

 \[
   \text{Rappel}=\frac{\text{Nombre de document pertinents récupérés}}{\text{Nombre total de documents pertinents}}
   \vspace{1cm}
 \]

 Un \textbf{rappel élevé} signifie que le système est capable de retrouver une
 grande partie des documents pertinents dans la base de données. Cependant, cela
 peut entraîner des résultats moins précis si le système retourne également des
 documents non pertinents.
 \subsection{F-mesure}
 \textbf{La F-mesure} (ou F-score) est une mesure combinée qui cherche à équilibrer la
 précision et le rappel. Elle est particulièrement utile lorsque ces deux
 critères sont en conflit. La F-mesure est la moyenne harmonique entre la
 précision et le rappel, ce qui lui permet de donner une évaluation unique tout
 en tenant compte des deux dimensions. La formule de la F-mesure est la suivante:
 \par

 \[
   \text{F-mesure} = 2 \times \frac{\text{Précision}\times\text{Rappel}}{\text{Précision}+\text{Rappel}}
   \vspace{1cm}
 \]

 La F-mesure est importante car elle fournit une vue d'ensemble de la
 performance du système, en s'assurant qu'un équilibre est trouvé entre
 récupérer des résultats pertinents (précision) et retrouver la majorité des
 résultats pertinents (rappel). Une \textbf{F-mesure élevée} indique un bon compromis
 entre les deux, ce qui est essentiel dans un système MIR où les utilisateurs
 veulent à la fois des résultats pertinents et une couverture maximale des
 éléments pertinents.

 \vfill
\end{section} 