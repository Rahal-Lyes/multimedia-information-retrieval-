\begin{document}

% Table des matières
\tableofcontents
\newpage

% Introduction avec ajout manuel dans la table des matières
\chapter*{Introduction}
\addcontentsline{toc}{chapter}{Introduction} % Ajoute l'introduction à la table des matières

\par 
Les ordinateurs, longtemps utilisés pour traiter des informations textuelles et numériques, jouent désormais un 
rôle central dans l'accès et la gestion d’informations multimédia. En effet, de nombreux domaines professionnels 
nécessitent des contenus non textuels pour répondre à des besoins spécifiques. Par exemple, les médecins consultent
 des radiographies, les architectes utilisent des plans de bâtiments, et les agents immobiliers montrent des 
 photographies de propriétés.
 \par Dans ces domaines, l’information visuelle est souvent aussi importante, voire plus, que le texte.
  Le besoin de recherche d’informations multimédia devient ainsi essentiel, car il permet d’accéder aux documents
   visuels ou audio nécessaires à ces professionnels. Il est difficile d’imaginer qu'une entreprise de construction 
   reçoive un plan de bâtiment uniquement sous forme textuelle, ou qu’un journal soit consulté sans la disposition 
   graphique de ses pages. La recherche d’informations multimédia devient aussi essentielle dans des contextes où 
   les documents sont principalement textuels mais nécessitent des annotations ou illustrations visuelles, comme 
   les formulaires d’assurance comportant des commentaires en marge.
   \par 
   Les progrès récents en matière de stockage et d’affichage numérique facilitent l’intégration de contenus multimédia
    dans les documents informatiques, et le traitement des images, vidéos, et sons devient plus accessible.
     La recherche d’informations multimédia permet aux utilisateurs de créer et de naviguer dans des bibliothèques
      de documents enrichis, où textes et médias se complètent pour offrir des perspectives variées et améliorer
       la qualité des informations accessibles.


% Contexte avec ajout manuel dans la table des matières
\section*{Contexte}
\addcontentsline{toc}{section}{Contexte} % Ajoute le contexte à la table des matières
Cette section permet de fournir un aperçu du contexte dans lequel cette étude s’inscrit...

% Objectifs avec ajout manuel dans la table des matières
\section*{Objectifs}
\addcontentsline{toc}{section}{Objectifs} % Ajoute les objectifs à la table des matières
Les principaux objectifs de ce projet sont les suivants :
\begin{itemize}
    \item Décrire le cadre théorique et les concepts de base ;
    \item Établir une méthodologie claire et structurée ;
    \item Fournir une analyse approfondie des résultats obtenus.
\end{itemize}

% Structure du document avec ajout manuel dans la table des matières
\section*{Structure du document}
\addcontentsline{toc}{section}{Structure du document} % Ajoute la structure du document à la table des matières
Ce document est structuré en plusieurs chapitres :
\begin{enumerate}
    \item Le chapitre 1 présente ...
    \item Le chapitre 2 se concentre sur ...
    \item Le chapitre 3 aborde ...
\end{enumerate}

\vfill

\end{document}
