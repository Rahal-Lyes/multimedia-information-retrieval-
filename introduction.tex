\begin{section}

 % Table des matières

 % Introduction avec ajout manuel dans la table des matières
 \chapter*{Introduction}
 \addcontentsline{toc}{chapter}{Introduction} % Ajoute l'introduction à la table des matières

 \par
 Les ordinateurs, longtemps utilisés pour traiter des informations textuelles et
 numériques, jouent désormais un rôle central dans l'accès et la gestion
 d’informations multimédia. En effet, de nombreux domaines professionnels
 nécessitent des contenus non textuels pour répondre à des besoins spécifiques.
 Par exemple, les médecins consultent des radiographies, les architectes
 utilisent des plans de bâtiments, et les agents immobiliers montrent des
 photographies de propriétés.
 \par
 Dans ces domaines, l’information visuelle est souvent aussi importante, voire
 plus, que le texte. Le besoin de recherche d’informations multimédia devient
 ainsi essentiel, car il permet d’accéder aux documents visuels ou audio
 nécessaires à ces professionnels. Il est difficile d’imaginer qu'une entreprise
 de construction reçoive un plan de bâtiment uniquement sous forme textuelle, ou
 qu’un journal soit consulté sans la disposition graphique de ses pages. La
 recherche d’informations multimédia devient aussi essentielle dans des
 contextes où les documents sont principalement textuels mais nécessitent des
 annotations ou illustrations visuelles, comme les formulaires d’assurance
 comportant des commentaires en marge.
 \par
 Les progrès récents en matière de stockage et d’affichage numérique facilitent
 l’intégration de contenus multimédia dans les documents informatiques, et le
 traitement des images, vidéos, et sons devient plus accessible. La recherche
 d’informations multimédia permet aux utilisateurs de créer et de naviguer dans
 des bibliothèques de documents enrichis, où textes et médias se complètent pour
 offrir des perspectives variées et améliorer la qualité des informations
 accessibles.

 \vfill

\end{section}
