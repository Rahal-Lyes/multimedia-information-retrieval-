
\begin{section}

\chapter{techniques de récupération basée sur le contenu (CBR)}
Les techniques de récupération basées sur le contenu permettent d’identifier,
de rechercher et de classer des éléments en analysant leurs caractéristiques
intrinsèques. Contrairement aux approches traditionnelles qui se fondent sur
des mots-clés ou des métadonnées (comme les tags ou descriptions), ces méthodes
analysent directement le contenu des éléments, qu’il s’agisse de données
visuelles, auditives ou textuelles.\par

Les \textbf{principales étapes} du processus incluent :

\begin{enumerate}
    \item \textbf{Extraction de caractéristiques} : Conversion des éléments multimédias en des représentations numériques compactes appelées descripteurs (par exemple, histogrammes de couleurs, spectrogrammes pour l’audio, etc.).
    \item \textbf{Indexation} : Organisation des descripteurs dans une structure de données adaptée (par exemple, arbre de recherche).
    \item \textbf{Recherche et comparaison} : Comparaison des descripteurs d’une requête avec ceux des éléments stockés en utilisant des mesures de similarité.
\end{enumerate}
\setcounter{section}{1}
\section{Applications pratiques}
\subsection{ Recherche d’images}
\textbf{Techniques utilisées :}
\begin{itemize}
    \item Histogrammes de couleurs : Mesures statistiques des couleurs dominantes d’une image.
    \item Textures : Identification des motifs répétitifs dans une image (par exemple, lignes, points).
    \item Formes : Analyse des contours et des structures géométriques.
    \item Deep Learning : Utilisation de réseaux de neurones convolutionnels (CNN) pour extraire des caractéristiques complexes.
\end{itemize}
\textbf{Exemple :} Recherche d’images similaires à partir d’un croquis ou d’une photo.

\subsection{Recherche audio}
\textbf{Techniques utilisées :}
\begin{itemize}
    \item Spectrogrammes : Représentation visuelle des fréquences dans un signal audio.
    \item Fingerprints audio : Création d’une empreinte unique basée sur des fréquences et des rythmes.
    \item Apprentissage automatique : Modèles d’apprentissage supervisé pour identifier des genres ou des instruments.
\end{itemize}
\textbf{Exemple :} Retrouver une chanson en fredonnant une mélodie (Shazam).

\subsection{ Recherche vidéo}
\textbf{Techniques utilisées :}
\begin{itemize}
    \item Analyse de frames : Extraction de caractéristiques à partir d’images individuelles dans une vidéo.
    \item Reconnaissance de scènes : Identification de contextes ou objets spécifiques dans une séquence.
    \item Indexation spatio-temporelle : Analyse des mouvements et changements dans le temps.
\end{itemize}
\textbf{Exemple :} Recherche de vidéos sportives contenant un type de but spécifique.

\subsection{ Recherche textuelle}
\textbf{Techniques utilisées :}
\begin{itemize}
    \item TF-IDF (Term Frequency-Inverse Document Frequency) : Évaluation de l’importance des mots dans un document.
    \item Word Embeddings : Représentation vectorielle des mots (ex. Word2Vec, GloVe).
    \item Recherche sémantique : Utilisation de modèles NLP pour comprendre le contexte.
\end{itemize}
\textbf{Exemple :} Rechercher des articles similaires à un texte donné.

\section{Méthodologies détaillées}
\subsection{ Extraction de caractéristiques}
Chaque type de contenu possède des descripteurs spécifiques :
\begin{itemize}
    \item \textbf{Images} : Histogrammes, gradients orientés (HOG), descripteurs locaux comme SIFT ou SURF.
    \item \textbf{Audio} : Coefficients cepstraux (MFCC), analyses FFT (Fast Fourier Transform).
    \item \textbf{Vidéo} : Fusion des caractéristiques spatiales (images) et temporelles (mouvements).
    \item \textbf{Texte} : Analyse syntaxique, fréquence des mots, bigrammes/trigrammes.
\end{itemize}

\subsection{Mesures de similarité}
Les mesures varient en fonction du type de données :
\begin{itemize}
    \item Distance Euclidienne : Mesure classique entre vecteurs.
    \item Cosine Similarity : Pour comparer des documents textuels.
    \item Intersection d’histogrammes : Adaptée aux données d’image.
    \item Distance de Levenshtein : Pour évaluer les différences entre chaînes de caractères.
\end{itemize}

\subsection{Apprentissage automatique}
Utilisation de techniques modernes pour améliorer la pertinence :
\begin{itemize}
    \item Clustering : Organisation non supervisée des données (ex. K-Means).
    \item Classification supervisée : Modèles comme les SVM ou les réseaux neuronaux pour catégoriser les données.
    \item Deep Learning : Approches complexes basées sur des réseaux de neurones profonds (CNN pour les images, RNN/LSTM pour les séquences).
\end{itemize}

\section{Avantages des CBR}
\begin{enumerate}
    \item Recherche basée sur le contenu réel plutôt que des étiquettes parfois erronées.
    \item Permet la recherche d'éléments visuels, sonores ou textuels avec une précision accrue.
    \item Exploitation des capacités d’intelligence artificielle pour s’adapter à des bases de données massives.
\end{enumerate}

\section{Défis et limites}
\begin{enumerate}
    \item Complexité calculatoire : Traitement et comparaison des descripteurs peuvent être coûteux pour de larges bases de données.
    \item Qualité des caractéristiques : Une mauvaise extraction peut nuire à la pertinence des résultats.
    \item Subjectivité : Les descripteurs ne capturent pas toujours la perception humaine (exemple : beauté subjective dans une image).
\end{enumerate}
\end{section}